Допустим, Виктор должен проверить, что общается с Пегги. Предварительно выполнены действия:

\begin{enumerate}
\item Трент сгенерировал пару ключей (открытый и закрытый);
\item Трент послал Виктору открытый ключ $\left\{v, n\right\}$, а Пегги - оба.
\end{enumerate}

Сама процедура проверки состоит в выполнении $T$ раз процедуры аккредитации:

\begin{enumerate}
\item Пегги выбирает случайное число $r$;
\item Пегги отправляет Виктору $m = r ^ 2 \ mod \ n$ ($n$ из открытого ключа);
\item Виктор отправляет Пегги случайный бит $b$;
\item Пегги, используя закрытый и открытый ключи, вычисляет $y$:
    \begin{itemize}
        \item если $b=0$: $y=r$;
        \item если $b=1$: $y= (r \cdot s) \ mod \ n$;
    \end{itemize}
\item Пегги посылает $y$ Виктору;
\item Виктор, используя открытый ключ Пегги, вычисляет $\bar{m}$:
    \begin{itemize}
        \item если $b=0$: $\bar{m}=r^2 \ mod \ n$;
        \item если $b=1$: $\bar{m}=(y^2 \cdot v)^2 \ mod \ n$;
    \end{itemize}
\item Виктор сравнивает $m$ и $\bar{m}$, которые должны быть равны для успешной аккредитации.

\end{enumerate}

Если все $T$ раз аккредитация проходит успешно, можно считать аутентификацию выполненной.
Вероятность ошибки составит $\frac{1}{2^T}$.